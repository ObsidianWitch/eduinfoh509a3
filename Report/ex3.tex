\section{Distances}

\noindent The goal of this 3rd exercise was to retrieve the distance between
each pair of authors.\\

First, we created a graph with nodes corresponding to each authors in the dblp
excerpt, and edges symbolizing direct co-authorship. The graph is undirected,
as such, if the \hl{<edge from="2" to="3"/>} edge is defined, then the
\hl{<edge from="3" to="2"/>} edge is not present in the xml representation of the
graph in order to avoid unecessary computation later.\\

\begin{lstlisting}[caption=Example of generated graph]
<?xml version="1.0"?>
<graph>
  <nodes>
    <node id="1" name="Mazeyar E. Makoui"/>
    <node id="2" name="Gunter Saake"/>
    <node id="3" name="Kai-Uwe Sattler"/>
    <node id="4" name="Andreas Heuer"/>
    ...
  </nodes>
  <edges>
    <edge from="2" to="3"/>
    <edge from="2" to="4"/>
    <edge from="3" to="4"/>
    ...
  </edges>
</graph>
\end{lstlisting}
\

From these nodes and edges we can then compute the distances by using a
recursive depth-first search algorithm \cite{cite:sodfs}. For each node, a
recursive depth-first search is run in order to retrieve the distance between
this author and all related authors.\\

The \hl{\$visited} sequence is used as a stack in order to avoid being stuck in
a cycle caused by nodes being revisited infinitely.\\

It also must be noted that once the DFS is finished for a node, the
\hl{ufn:filter\_duplicates(\$node, \$distances)} function is ran on the computed
distances for the current node. Duplicates are possible in the
(author1, author2) pairs because there are many paths that can go from the
current author to another author. So, we filter results by selecting the minimal
distance from the duplicate pairs.\\

\begin{framewarning}
    The way we handled the computation is probably not the most efficient one,
    since for the given excerpt of dblp, for the program running on one core at
    2.60 GHz, it took approximately 7 minutes. For the full dblp bibliography
    containing more than 2 millions nodes, it would take years. The computation
    time could be divided by executing parts of the computation on multiple CPU
    cores and on multiple computers.
\end{framewarning}

\newpage
